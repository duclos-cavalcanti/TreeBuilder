\documentclass{article}

\usepackage[a4paper,top=2.5cm,bottom=2cm,left=3cm,right=3cm,marginparwidth=1.75cm]{geometry}
\usepackage{graphicx}
\usepackage{fancyhdr}

\fancyhead[L]{M.Sc. Thesis - Kick-Off Document}
\fancyhead[C]{}
\fancyhead[R]{\includegraphics[width=10mm, height=5mm]{tum_logo.png}}

\pagestyle{fancy}
\makeatletter
\renewcommand{\maketitle}{%
    \thispagestyle{fancy}%
    \begin{center}
        \Large\bfseries\@title
    \end{center}
    \vspace{-0.25cm}
    \begin{center}
        Daniel Duclos-Cavalcanti (Student),
        \\
        Muhammad Haseeb (NYU Supervisor),
        ~Navidreza Asadi (TUM Supervisor) 
        \\Chair of Communication Networks, School of CIT
        \\Technical University of Munich (TUM)
    \end{center}
}
\makeatother

\title{VM Selection for Financial Exchanges in the Cloud}

\begin{document}

\maketitle

\section{Overview}

% Lemondrop uses an application's request pattern and real-time intra-VM OWD latency to 
% frame the problem of node-to-VM selection as Quadratic Assignment Problem. LemonDrop implements an
% optimization method obtain a locally optimal solution to a relaxed version of the problem.

Financial exchanges consider a migration to the cloud for scalability, robustness, and cost-efficiency.
Jasper \cite{haseeb2024jasper} presents a scalable and fair multicast solution for cloud-based exchanges, 
addressing the lack of cloud-native mechanisms for such. 
To achieve this, Jasper employs an overlay multicast tree, leveraging clock synchronization, kernel-bypass techniques, 
and more.
However, there are opportunities for enhancement by confronting the issue of inconsistent VM performance 
within identical instances. LemonDrop \cite{sachidananda2022scheduling} tackles this problem,
selecting and scheduling a subset of VMs optimized for an application's latency needs by detecting under-performing VMs. 
% It does so by using an application's request pattern and real-time intra-VM OWD latency measurements to 
% frame the problem as a relaxed Quadratic Assignment Problem (QAP). 
Jasper's performance can drastically be affected by under-performing VMs.
Yet, we believe that LemonDrop's approach of using all-to-all latency measurements and an optimization routine 
for the framed QAP is an overly complex solution and may be disproportionally robust 
for Jasper's use-case of a multicast-tree. 
% Moreover, it may capture an unrealistic snapshot of the cluster's behavior due to 
% the known high latency variance in the cloud.
Finally, the proposed work aims to develop an even simpler heuristic, that achieves reasonably good results 
adapted to the current smaller problem set.

\section{Objectives}
Develop a VM selection heuristic for tree-like networks in the cloud.
\begin{enumerate}
    \item Implement a Server-Client Manager application:
    \begin{enumerate}
        \item Server: Allows user to \textbf{run}, \textbf{terminate} and \textbf{report} on processes across a cluster.        
        \begin{enumerate}
            \item Connects to client nodes (VMs) and issues action-requests:
            \begin{enumerate}
                \item Action(A): Launch a process/program, store and report PID.
                \item Action(B): Report information on ongoing process.
                \item Action(C): Kill a previously ran process.
            \end{enumerate}
        \end{enumerate}
        \item Client: Waits on Server's connection and requests.
    \end{enumerate}
    \item Employ Testbench via Server-Client Manager:
    \begin{enumerate}
        \item Server allocates a pool of \textbf{N} VMs, runs and terminates Jasper on initial configuration.
        \item Server applies \textbf{heuristic} to develop a new tree configuration by iterativally:
            \begin{enumerate}
                \item Running/Obtaining reports on intra-VM latency among nodes in the pool.
                \item Selecting/Assigning best node to the next available slot in the tree.
                \item Removing selected node from option-pool.
            \end{enumerate}
        \item Server re-deploys new configuration.
    \end{enumerate}
\end{enumerate}

\section{Experimental Setup}
Cloud VM instances of \textit{c2d-highcpu-8} type would be deployed on Google Cloud's Platform. 
Each machine offers 8 virtual AMD Milan CPU's, 16GB of Memory and 16Gbps of Network Bandwidth. 
Results would be compared among a vanilla Jasper run, a heuristic-proposed 
and a LemonDrop proposed configuration of Jasper, as well as done so across different tree sizes.

\bibliographystyle{IEEEtran}
\bibliography{references}    

\end{document}

