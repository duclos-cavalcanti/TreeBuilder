\documentclass{article}

\usepackage[a4paper,top=2.5cm,bottom=2cm,left=3cm,right=3cm,marginparwidth=1.75cm]{geometry}
\usepackage{graphicx}
\usepackage{fancyhdr}

\fancyhead[L]{M.Sc. Thesis - Kick-Off Document}
\fancyhead[C]{}
\fancyhead[R]{\includegraphics[width=10mm, height=5mm]{tum_logo.png}}

\pagestyle{fancy}
\makeatletter
\renewcommand{\maketitle}{%
    \thispagestyle{fancy}%
    \begin{center}
        \Large\bfseries\@title
    \end{center}
    \vspace{-0.25cm}
    \begin{center}
        Daniel Duclos-Cavalcanti (Student),
        \\
        Muhammad Haseeb (NYU Supervisor),
        ~Navidreza Asadi (TUM Supervisor) 
        \\Chair of Communication Networks, School of CIT
        \\Technical University of Munich (TUM)
    \end{center}
}
\makeatother

\title{VM Selection for Financial Exchanges in the Cloud}

\begin{document}

\maketitle

\section{Overview}

Financial exchanges have shown interest in migrating their current infrastructure 
to the public cloud. The benefits are agreed upon by both the industry and 
academia, promising a more scalable, robust, and cost-efficient infrastructure. 
However, a vital concern is the lack of support for fair and performant multicast 
in the public cloud. Exchanges need to disseminate market data to market participants 
(MPs) both fast and fairly. Every MP has to receive market state updates 
almost simultaneously to not create an unfair advantage among MPs.

Different than the current exchange's on-premise data centers, the cloud does not 
offer native mechanisms for fair data delivery. Recent work, namely Jasper \cite{haseeb2024jasper}, 
has addressed this gap, offering a solution that creates an overlay multicast tree, leveraging 
up-to-date advancements in clock synchronization, kernel by-passing, and hedging, 
to present a scalable and fair multicast on the cloud. Jasper offers a commendable 
alternative, outperforming contemporary efforts, and Amazon's in-house multicast solution, 
as well as addressing known irregularities regarding network latency in the cloud.

However, there are possible avenues for improvement. LemonDrop \cite{sachidananda2022scheduling}, a component 
of a larger body of work, tackles the real issue of inconsistent VM performance within identical instance configurations 
in the cloud. Under-performing VMs within a given instance class are called \textit{Lemons}. 
LemonDrop was developed to select and schedule a subset of VMs optimized for a given application's latency needs, by quickly 
detecting lemon VMs, repositioning them across the system, or dropping them completely.
It does so by framing the selection and scheduling of VMs as a Quadratic Assignment Problem, where 
traffic flow between facilities, each assigned to a location, is to be minimized.
LemonDrop treats services within an application as facilities and the VMs themselves as the locations. 

% LemonDrop takes as inputs the one-way delay (OWD) matrix of a given cloud cluster and 
% the given application's load-matrix. Finally, it approximately solves the framed QAP via an 
% algorithm called Fast Approximate Quadratic Programming and further optimization methods to relax constraints imposed 
% in this scenario.

Within the context of Jasper, lemons have the potential to drastically affect the overall system's performance.
Inspired by LemonDrop's VM selection method, the proposed work here aims to develop a simpler heuristic that 
can achieve reasonably good results adapted to the smaller problem set of a multicast tree. 
Therefore, significant improvements could be brought to Jasper's deployment and performance as a 
cloud tenant solution for financial exchanges in the cloud.

% \begin{itemize}
%     \item What problem do you want to solve? What is the main question(s)?
%     \item Why is it important?
%     \item What is the status quo?
%     \item How do you approach the problem?
%     \item What are the potential challengesa
% \end{itemize}
 
\section{Objectives}
\begin{enumerate}
    \item Develop a VM selection heuristic for the Cloud. 
    \item Adapt heuristic to Jasper's tree-like network structure.
\end{enumerate}

\section{Experimental Setup}
Creation of a cloud stack and data analysis are be done locally. The cloud stack deployment, node benchmarking and
heuristic development would be done via Google Cloud credits provided by Dr.Sivaraman's and his team at Systems@NYU.

\bibliographystyle{IEEEtran}
\bibliography{references}    
    
\end{document}
