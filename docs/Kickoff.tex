\documentclass{article}

\usepackage[a4paper,top=2.5cm,bottom=2cm,left=3cm,right=3cm,marginparwidth=1.75cm]{geometry}
\usepackage{graphicx}
\usepackage{fancyhdr}

\fancyhead[L]{M.Sc. Thesis - Kick-Off Document}
\fancyhead[C]{}
\fancyhead[R]{\includegraphics[width=10mm, height=5mm]{tum_logo.png}}

\pagestyle{fancy}
\makeatletter
\renewcommand{\maketitle}{%
    \thispagestyle{fancy}%
    \begin{center}
        \Large\bfseries\@title
    \end{center}
    \vspace{-0.25cm}
    \begin{center}
        Daniel Duclos-Cavalcanti (Student),
        \\
        Muhammad Haseeb (NYU Supervisor),
        ~Navidreza Asadi (TUM Supervisor) 
        \\Chair of Communication Networks, School of CIT
        \\Technical University of Munich (TUM)
    \end{center}
}
\makeatother

\title{VM Selection for Financial Exchanges in the Cloud}

\begin{document}

\maketitle

\section{Overview}

Financial exchanges have shown interest in migrating their current infrastructure 
to the public cloud, potentially allowing for greater scalability, robustness, and lower costs.
In contrast to the exchange's current on-premise data centers, the public cloud does not 
offer native mechanisms for fair and performant multicast delivery. Disseminating market data 
to market participants (MPs) both fast and almost simultaneously is a needed requirement to ensure a fair market.

Recent work, namely Jasper \cite{haseeb2024jasper}, addressed this by presenting a scalable and fair multicast 
solution for financial exchanges in the cloud. It does so via the employment of an overlay multicast tree, 
clock synchronization, kernel by-passing, and more. Jasper achieves considerable performance and fairness, 
and attempts to address the high network-latency variance in the cloud through custom techniques such as hedging. 
However, there is room for improvement by optimizing in-host latency via confronting the issue of inconsistent VM performance in the cloud.

LemonDrop \cite{sachidananda2022scheduling} tackles this very problem of discrepant VM behavior 
within identical instance configurations in the cloud. LemonDrop selects and schedules 
a subset of VMs optimized for an application's latency needs, by quickly detecting under-performing ones (\textit{Lemons or Stragglers}).
It frames the selection and scheduling of VMs as a Quadratic Assignment Problem (QAP), minimizing requests among VMs
by reconfiguring their relative placement and dropping individual ones completely. 
To accomplish this, LemonDrop uses knowledge of an application's request patterns and real-time intra-VM latency to obtain its proposed configuration.

% LemonDrop \cite{sachidananda2022scheduling} tackles this very problem of inconsistent VM performance 
% within identical instance configurations in the cloud. LemonDrop was developed to select and schedule 
% a subset of VMs optimized for a given application's latency needs, by quickly detecting under-performing VMs (\textit{Lemons or Stragglers}).
% It does so by framing the selection and scheduling of VMs as a Quadratic Assignment Problem (QAP), where 
% traffic flow between facilities, each assigned to a location, is to be minimized. 
% LemonDrop treats services within an application as facilities and the VMs themselves as the locations. 

Stragglers can likely affect Jasper's overall system performance drastically.
This proposed work aims to develop a simpler heuristic than LemonDrop's, that achieves reasonably good results 
adapted to the smaller problem set of a multicast tree. 
We believe that due to the known high latency variance in the cloud, LemonDrop's approach of using all-to-all latency measurements
is an overly-complex solution that may capture an unrealistic snapshot of the cluster's behavior.

% Therefore, significant improvements could be brought to Jasper's deployment and performance as a 
% modern solution for financial exchanges in the cloud.

\section{Objectives}
Develop a VM Selection Heuristic for tree-like networks in the cloud.
\begin{enumerate}
    \item Develop Server-Client Manager application.
    \begin{enumerate}
        \item Server: Connects to all client nodes and applies requests onto them:
        \begin{enumerate}
            \item Launch a process and store it's PID.
            \item Report data on a previously ran process.
            \item Kill a previously ran process.
        \end{enumerate}
        \item Client: Waits on Server's connection and reacts to requests.
    \end{enumerate}
    \item Apply Testbench through Server-Client Manager:
    \begin{enumerate}
        \item Server allocate a pool of `N` VMs in the cloud and stores initial configuration.
        \item Server is able to command client nodes to run, terminate and report on:
        \begin{enumerate}
            \item Jasper
            \item Intra-VM OWD Latency Measurements
            \item LemonDrop
        \end{enumerate}
        \item Server can change change cluter configuration via killing and relaunching processes.
    \end{enumerate}
    \item Develop Heuristic:
    \begin{enumerate}
        \item Recreate tree by iterativally:
        \begin{enumerate}
            \item Benchmark intra-VM latency and choose best node among the pool.
            \item Place/Assign node to the next available slot in the tree.
        \end{enumerate}
    \end{enumerate}
\end{enumerate}

\section{Experimental Setup}
Cloud stack deployment, node benchmarking and heuristic development/formulation would be done 
via Google Cloud Platform (GCP) credits provided by Dr.Sivaraman's and his team at Systems@NYU.

\bibliographystyle{IEEEtran}
\bibliography{references}    

\end{document}

