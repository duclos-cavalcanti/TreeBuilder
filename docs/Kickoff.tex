\documentclass{article}

\usepackage[a4paper,top=2.5cm,bottom=2cm,left=3cm,right=3cm,marginparwidth=1.75cm]{geometry}
\usepackage{graphicx}
\usepackage{fancyhdr}

\fancyhead[L]{M.Sc. Thesis - Kick-Off Document}
\fancyhead[C]{}
\fancyhead[R]{\includegraphics[width=10mm, height=5mm]{tum_logo.png}}

\pagestyle{fancy}
\makeatletter
\renewcommand{\maketitle}{%
    \thispagestyle{fancy}%
    \begin{center}
        \Large\bfseries\@title
    \end{center}
    \vspace{-0.25cm}
    \begin{center}
        Daniel Duclos-Cavalcanti (Student),
        \\
        Muhammad Haseeb (NYU Supervisor),
        ~Navidreza Asadi (TUM Supervisor) 
        \\Chair of Communication Networks, School of CIT
        \\Technical University of Munich (TUM)
    \end{center}
}
\makeatother

\title{VM Selection for Financial Exchanges in the Cloud}

\begin{document}

\maketitle

\section{Overview}

Financial exchanges have shown interest in migrating their current infrastructure 
to the public cloud. The benefits are agreed upon by both the industry and 
academia, promising a more scalable, robust, and cost-efficient infrastructure. 
However, in contrast to the exchange's on-premise data centers, the public cloud does not 
currently offer native mechanisms for fair and performant data delivery. 
Exchanges need to disseminate market data to market participants (MPs) both fast and 
virtually simultaneously to not create unfair advantages among MPs.

Recent work, namely Jasper \cite{haseeb2024jasper}, addressed this by presenting a scalable and fair multicast 
for financial exchanges in the cloud. It does so via the employment of an overlay multicast tree and leveraging 
up-to-date advancements in clock synchronization, kernel by-passing, and hedging to simultaneously 
achieve considerable performance and fairness. Jasper offers a commendable alternative, 
outperforming a previous system CloudEx and Amazon's commercial multicast solution.

Moreover, LemonDrop \cite{sachidananda2022scheduling} tackles the real issue of 
inconsistent VM performance within identical instance configurations in the cloud. 
LemonDrop was developed to select and schedule a subset of VMs optimized for a given application's latency needs, by quickly 
detecting under-performing VMs (\textit{Lemons or Stragglers}).
It does so by framing the selection and scheduling of VMs as a Quadratic Assignment Problem (QAP), where 
traffic flow between facilities, each assigned to a location, is to be minimized. 
LemonDrop treats services within an application as facilities and the VMs themselves as the locations. 

% LemonDrop takes as inputs the one-way delay (OWD) matrix of a given cloud cluster and 
% the given application's load-matrix. Finally, it approximately solves the framed QAP via an 
% algorithm called Fast Approximate Quadratic Programming and further optimization methods to relax constraints imposed 
% in this scenario.

Straggler VMs have the potential to drastically affect Jasper's overall system performance.
Inspired by LemonDrop's VM selection method, the proposed work here aims to develop a simpler heuristic that 
can achieve reasonably good results adapted to the smaller problem set of a multicast tree. 
Therefore, significant improvements could be brought to Jasper's deployment and performance as a 
modern solution for financial exchanges in the cloud.

\section{Objectives}
\begin{enumerate}
    \item Develop a VM Selection Heuristic for the Cloud.
\end{enumerate}

\section{Experimental Setup}
Cloud stack deployment, node benchmarking and heuristic development/formulation would be done 
via Google Cloud Platform (GCP) credits provided by Dr.Sivaraman's and his team at Systems@NYU.

\begin{enumerate}
    \item Allocate a pool of VMs on the cloud.
    \item Iterativally:
    \begin{enumerate}
        \item Benchmark intra-VM latency and choose best node among the pool.
        \item Place/Assign node to the next available slot in the tree.
    \end{enumerate}
    \item Conduct Jasper-runs on final tree configuration.
    \item Compare to vanilla Jasper deployment.
\end{enumerate}

\bibliographystyle{IEEEtran}
\bibliography{references}    

% \begin{itemize}
%     \item What problem do you want to solve? What is the main question(s)?
%     \item Why is it important?
%     \item What is the status quo?
%     \item How do you approach the problem?
%     \item What are the potential challengesa
% \end{itemize}
 
    
\end{document}

