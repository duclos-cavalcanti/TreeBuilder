\documentclass{article}

\usepackage[a4paper,top=2.5cm,bottom=2cm,left=3cm,right=3cm,marginparwidth=1.75cm]{geometry}
\usepackage{graphicx}
\usepackage{fancyhdr}

\fancyhead[L]{M.Sc. Thesis - Kick-Off Document}
\fancyhead[C]{}
\fancyhead[R]{\includegraphics[width=10mm, height=5mm]{tum_logo.png}}

\pagestyle{fancy}
\makeatletter
\renewcommand{\maketitle}{%
    \thispagestyle{fancy}%
    \begin{center}
        \Large\bfseries\@title
    \end{center}
    \vspace{-0.25cm}
    \begin{center}
        Daniel Duclos-Cavalcanti (Student),
        \\
        Muhammad Haseeb (NYU Supervisor),
        ~Navidreza Asadi (TUM Supervisor) 
        \\Chair of Communication Networks, School of CIT
        \\Technical University of Munich (TUM)
    \end{center}
}
\makeatother

\title{VM Selection for Financial Exchanges in the Cloud}

\begin{document}

\maketitle

\section{Overview}

% Lemondrop uses an application's request pattern and real-time intra-VM OWD latency to 
% frame the problem of node-to-VM selection as Quadratic Assignment Problem. LemonDrop implements an
% optimization method obtain a locally optimal solution to a relaxed version of the problem.

Financial exchanges consider a migration to the cloud for scalability, robustness, and cost-efficiency.
Jasper \cite{haseeb2024jasper} presents a scalable and fair multicast solution for cloud-based exchanges, 
addressing the lack of cloud-native mechanisms for such. 
To achieve this, Jasper employs an overlay multicast tree, leveraging clock synchronization, kernel-bypass techniques, 
and more.
However, there are opportunities for enhancement by confronting the issue of inconsistent VM performance 
within identical instances. LemonDrop \cite{sachidananda2022scheduling} tackles this problem, detecting under-performing VMs in a cluster 
and selecting a subset of VMs optimized for a given application's latency needs.
% It does so by using an application's request pattern and real-time intra-VM OWD latency measurements to 
% frame the problem as a relaxed Quadratic Assignment Problem (QAP). 
% Jasper's performance can potentially be improved by employing a VM selection method.
Yet, we believe that LemonDrop's approach of using time-expensive all-to-all latency measurements and an optimization routine 
for the framed Quadratic Assignment Problem (QAP) is overly complex. 
% for the there framed Quadratic Assignment Problem (QAP) is an overly complex solution. 
% It is disproportionally robust and  unscalable for Jasper's use case of a multicast-tree.
% Moreover, it may capture an unrealistic snapshot of the cluster's behavior due to 
% the known high latency variance in the cloud.
The proposed work aims to develop a simpler and scalable heuristic, that achieves reasonably good results
within Jasper's time constraints. 
% We expect to encounter challenges in being able to reach considerable 
% improvements under such tight latency windows.

\section{Objectives}
Develop a VM selection heuristic for Financial Exchanges in the cloud.
\begin{enumerate}
    \item Implement a Server-Client Manager application:
    \begin{enumerate}
        \item Server: Allows user to \textbf{run}, \textbf{terminate} and \textbf{report} on processes across a cluster.        
        \begin{enumerate}
            \item Connects to client nodes (VMs) and issues action-requests:
            \begin{enumerate}
                \item Action(A): Launch a process/program, store and report PID.
                \item Action(B): Report information on ongoing process.
                \item Action(C): Kill a previously ran process.
            \end{enumerate}
        \end{enumerate}
        \item Client: Waits on Server's connection and requests.
    \end{enumerate}
    \item Develop Testbench Framework via Server-Client Manager:
    \begin{enumerate}
        \item Server allocates \textbf{N} VMs, then runs and terminates Jasper on initial configuration.
        \item Server applies \textbf{Heuristic} to produce new tree configuration.
        \item Server re-deploys Jasper on new configuration.
    \end{enumerate}
    \item Heuristic: Choose \textbf{K} VMs, apply selection method for \textbf{F} nodes, assign them to next layer.
        \begin{enumerate}
            \item \textbf{K} VMs Selection: Initially random, we intend on expanding.
            \item \textbf{F} VMs Selection: Analyzing reports on intra-VM OWD latency among the \textbf{K} nodes.
        \end{enumerate}
\end{enumerate}

\section{Experimental Setup}
Cloud VM instances of \textit{c2d-highcpu-8} type would be deployed on Google Cloud's Platform. 
Each machine offers 8 virtual AMD Milan CPU's, 16GB of Memory and 16Gbps of Network Bandwidth. 
Results would be compared among a vanilla Jasper run, a heuristic-proposed 
and a LemonDrop proposed configuration of Jasper. Experiments would be run across 
two pre-selected tree structures, chosen based on a realistic cloud exchange scenario 
and Jasper's best fitting configurations for them.

\bibliographystyle{IEEEtran}
\bibliography{references}    

\end{document}

